% Created 2020-10-12 Mon 08:57
% Intended LaTeX compiler: pdflatex
\documentclass[11pt]{article}
\usepackage[utf8]{inputenc}
\usepackage[T1]{fontenc}
\usepackage{graphicx}
\usepackage{grffile}
\usepackage{longtable}
\usepackage{wrapfig}
\usepackage{rotating}
\usepackage[normalem]{ulem}
\usepackage{amsmath}
\usepackage{textcomp}
\usepackage{amssymb}
\usepackage{capt-of}
\usepackage{hyperref}
\author{Sai Ashirwad R}
\date{\today}
\title{Taylor vs Fourier}
\hypersetup{
 pdfauthor={Sai Ashirwad R},
 pdftitle={Taylor vs Fourier},
 pdfkeywords={},
 pdfsubject={},
 pdfcreator={Emacs 26.3 (Org mode 9.4)}, 
 pdflang={English}}
\begin{document}

\maketitle
\tableofcontents

\href{http://dev.ipol.im/\~coco/website/taylorfourier.html\#:\~:text=When\%20x\%20and\%20\%CE\%B8\%20are,a\%20sum\%20of\%20sinusoidal\%20waves.}{Taylor and fourier are the same}

\begin{itemize}
\item taylor series
\end{itemize}
$$
f(x)=c_0+c_1x+c_2x^2+c_3x^3+\cdots
$$

\begin{itemize}
\item Fourier series
\end{itemize}
$$
g(\theta)=c_0+c_1\exp(i\theta)+c_2\exp(2i\theta)+c_3\exp(3i\theta)+\cdots
$$


\begin{itemize}
\item they are the same thing in complex numbers
\end{itemize}
$$
f(z)=c_0+c_1z+c_2z^2+c_3z^3+\cdots
$$

\begin{itemize}
\item restrict z to real axis, i.e. z = x, obtain Taylor series of f

\item restrict z to unit circle \(z = e^{i\theta}\) - obtain fourier series of function \(g(\theta) = f(e^{i\theta})\)

\item \url{https://math.stackexchange.com/questions/7301/connection-between-fourier-transform-and-taylor-series}
\item \begin{quote}
\url{https://math.stackexchange.com/a/7380}
There is an analogy, more direct for fourier series. Both Fourier series and Taylor series are decompositions of a function f(x), which is represented as a linear combination of a (countable) set of functions. The function is then fully specified by a sequence of coefficients, instead of by its values f(x) for each x. In this sense, both can be called a transform f(x)↔\{a0,a1,\ldots{}\}

For the Taylor series (around 0, for simplicity), the set of functions is \{1,x,x2,x3\ldots{}\}. For the Fourier series is \{1,sin(ωx),cos(ωx),sin(2ωx),cos(2ωx)\ldots{}\}.

Actually the Fourier series is one the many transformations that uses an orthonomal basis of functions. It is shown that, in that case, the coefficients are obtained by ``projecting'' f(x) onto each basis function, which amounts to an inner product, which (in the real scalar case) amounts to an integral. This implies that the coefficients depends on a global property of the function (over the full ``period'' of the function).

The Taylor series (which does not use a orthonormal basis) is conceptually very different, in that the coeffients depends only in local properties of the function, i.e., its behaviour in a neighbourhood (its derivatives).
\end{quote}
\end{itemize}
\end{document}
